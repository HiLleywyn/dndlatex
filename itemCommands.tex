% Custom command definitions for D&D item cards
% Requires the xparse package for \NewDocumentCommand macros

\newcommand{\ItemTags}[4]{% #1=item type, #2=rarity, #3=attunement type (0=none, 1=attunement, 2=special), #4=special attunement requirement
    \ifnum#3=0
        \textbf{#1, #2 }
        \\[0.1ex]

    \fi

    \ifnum#3=1
        \textbf{#1, #2 (requires attunement)}
        \\[0.1ex]

    \fi

    \ifnum#3=2
        \textbf{#1, #2 (requires attunement by a #4)}
        \\[0.1ex]

    \fi


}

\newcommand{\FlavorText}[2]{%#1=1 for flavor text, #2=flavor text
    \ifnum#1=1
        #2
        \\[0.1ex]

    \fi
}

\newcommand{\ChargeMechanics}[9]{% #1=total charges, #2=dice num, #3=dice sides, #4=modifier, #5=reset time, #6=1 for special charge condition, #7=special charge condition, #8=1 for last charge effect, #9=last charge effect
    \ifnum#1>1
        This item has #1 charges and regains #2d#3 + #4 expended charges daily at #5.
        \ifnum#6=1
            #7
        \fi
        \ifnum#8=1
            #9
        \fi
        \\[0.25ex]

    \fi
    \ifnum#1=1
        This item has #1 charge and regains #2d#3 + #4 expended charges daily at #5.
        \ifnum#6=1
            #7
        \fi
        \ifnum#8=1
            #9
        \fi
        \\[0.25ex]

    \fi

}

\newcommand{\SpellCasting}[6]{% #1=1 for spellcasting, #2=spell bonus, #3=spell bonus condition, #4=1 for additional spell effects, #5=spells, #6=additional spell effects
  \ifnum#1=1
    \textbf{Spells. }
    \ifnum#2>0
        You gain a +#2 bonus to your spell #3
    \fi
    #5
    \ifnum#4=1
        #6
    \fi
    \\[0.25ex]

  \fi

}
\newcommand{\CurseText}[2]{% #1=1 for curse, #2 for curse text
  \ifnum#1=1
    \textbf{Curse.} #2
    \\[0.25ex]

  \fi
}
\newcommand{\BonusText}[3]{% #1=1 for bonus text, #2 for title, #3 for text
  \ifnum#1=1
    \textbf{#2.} #3
    \\[0.1ex]

  \fi
}

% Weapon statistics table
\NewDocumentCommand{\WeaponStatBlock}{m m m}{%
  \begin{center}
    \begin{tabular}{ccc}
      \textbf{Damage} & \textbf{Weight} & \textbf{Properties}\\
      #1 & #2 & #3
    \end{tabular}
  \end{center}
  \vspace{0.5ex}
}

% Armor statistics table
\NewDocumentCommand{\ArmorStatBlock}{m m m m}{%
  \begin{center}
    \begin{tabular}{cccc}
      \textbf{AC} & \textbf{Strength} & \textbf{Stealth} & \textbf{Weight}\\
      #1 & #2 & #3 & #4
    \end{tabular}
  \end{center}
  \vspace{0.5ex}
}

% Simplified creature or NPC stat block
\NewDocumentCommand{\CreatureStats}{m m m m m m m m m m}{%
  \begin{center}
    \begin{tabular}{cccc}
      \textbf{AC} & \textbf{HP} & \textbf{Speed} & \textbf{CR}\\
      #1 & #2 & #3 & #10
    \end{tabular}\\[1ex]
    \begin{tabular}{cccccc}
      \textbf{STR} & \textbf{DEX} & \textbf{CON} & \textbf{INT} & \textbf{WIS} & \textbf{CHA}\\
      #4 & #5 & #6 & #7 & #8 & #9
    \end{tabular}
  \end{center}
  \vspace{0.5ex}
}

% Schools are:
%Abjuration, Conjuration, Divination, Enchantment, Evocation, Illusion, Necromancy, Transmutation
\newcommand{\SpellCardText}[8]{% #1=Spell name, #2=level[10 to turn off], #3=school, #4=casting time, #5=Range, #6=duration, #7={1-concentration, 0-not}, #8=Components
  \textbf{#1} \hfill
  \ifnum#2<10
  {\itshape
    \ifnum#2=0
      #3 Cantrip\\
    \else
        \ifnum#2=1
        1\textsuperscript{st}
        \fi
        \ifnum#2=2
        2\textsuperscript{nd}
        \fi
        \ifnum#2=3
        3\textsuperscript{rd}
        \fi
        \ifnum#2>3
        #2\textsuperscript{th}
        \fi
        level #3\\
    \fi
  \fi}
  \textbf{Casting Time: }#4 \hfill \textbf{Range: }#5\\
  \textbf{Duration: } {\ifnum#7=1  Concentration,  \fi} #6\\
  \textbf{Components: } {#8}\\[2ex]
}

% Generic property table for uniform formatting across cards
\NewDocumentCommand{\CardPropertyBlock}{O{--} O{--} O{--} O{--} O{--}}{% attack, defense, school, duration, ac
  \begin{center}
    \begin{tabular}{@{}ll@{}}
      \textbf{Attack} & #1\\
      \textbf{Defense} & #2\\
      \textbf{Spell School} & #3\\
      \textbf{Duration} & #4\\
      \textbf{AC} & #5\\
    \end{tabular}
  \end{center}
  \vspace{0.5ex}
}
% \begin{itemcard}{Item Name}
%     \begin{center}
%       \begin{tcolorbox}[width=\linewidth, boxrule=3.0pt, colframe=black, colback=white, boxsep=0pt, top=0pt, bottom=0pt, left=0pt, right=0pt, center={yshift=-5mm}]

%         \includegraphics[width=\linewidth]{item_pics/item.png}
%         % Replace with your image file name %

%       \end{tcolorbox}
%     \end{center}
%     \begin{lowerpart}
%         % Any where you see {___ Flag}, replacing it a {1} turns on that command.

%         % \ItemTags{Type}{Rarity}{Attunement Type}{Special Attunement Requirement}
%         % {Attunement Type} = {0}, No Attunement
%         % {Attunement Type} = {1}, Attunement
%         % {Attunement Type} = {2}, Attunement with conditions

%         % \FlavorText{Flavor Flag}{Flavor text placeholder}

%         % \ChargeMechanics{Total # of Charges}{# of Dice}{# of Sides}{Modifier}{Reset Time}{Special Recharge Flag}{Special Recharge Text}{Last Charge Effect Flag}{Last Charge Effect Text}

%         % \SpellCasting{Spellcasting Flag}{Spell Bonus}{Spell Bonus Condition}{Additional Effect Flag}{Spells List}{Additional Spell Effects}
%         % \BonusText{Bonus Flag}{Bonus Title}{Bonus Text}
%         % \CurseText{Curse Flag}{Curse Description}


%         \ItemTags{Type (subtype)}{rarity}{2}{creature}
%         \FlavorText{1}{You can do multiple cards of different sizes in one go in case you want to print them off.}
%         \CurseText{1}{It doesn't make separate PNGs for each item card.
%         It also doesn't allow for giving each card a different color scheme.}
%     \end{lowerpart}
% \end{itemcard}

% \end{document}
